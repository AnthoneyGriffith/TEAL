%
\documentclass[pdf,12pt]{article}

%\usepackage{times}
%\usepackage[FIGBOTCAP,normal,bf,tight]{subfigure}
\usepackage{amsmath}
\usepackage{amssymb}
%\usepackage{pifont}
\usepackage{enumerate}
\usepackage{listings}
\usepackage{fullpage}
\usepackage{xcolor}          % Using xcolor for more robust color specification
%\usepackage{ifthen}          % For simple checking in newcommand blocks
%\usepackage{textcomp}
%\usepackage{authblk}         % For making the author list look prettier
%\renewcommand\Authsep{,~\,}

% Custom colors
\definecolor{deepblue}{rgb}{0,0,0.5}
\definecolor{deepred}{rgb}{0.6,0,0}
\definecolor{deepgreen}{rgb}{0,0.5,0}
\definecolor{forestgreen}{RGB}{34,139,34}
\definecolor{orangered}{RGB}{239,134,64}
\definecolor{darkblue}{rgb}{0.0,0.0,0.6}
\definecolor{gray}{rgb}{0.4,0.4,0.4}

\lstset {
  basicstyle=\ttfamily,
  frame=single
}

\lstdefinestyle{XML} {
    language=XML,
    extendedchars=true,
    breaklines=true,
    breakatwhitespace=true,
%    emph={name,dim,interactive,overwrite},
    emphstyle=\color{red},
    basicstyle=\ttfamily,
%    columns=fullflexible,
    commentstyle=\color{gray}\upshape,
    morestring=[b]",
    morecomment=[s]{<?}{?>},
    morecomment=[s][\color{forestgreen}]{<!--}{-->},
    keywordstyle=\color{cyan},
    stringstyle=\ttfamily\color{black},
    tagstyle=\color{darkblue}\bf\ttfamily,
    morekeywords={name,type},
%    morekeywords={name,attribute,source,variables,version,type,release,x,z,y,xlabel,ylabel,how,text,param1,param2,color,label},
}
\lstset{language=xml}

\usepackage{titlesec}
\newcommand{\sectionbreak}{\clearpage}
\setcounter{secnumdepth}{4}


%%%%%%%% Begin comands definition to input python code into document
\usepackage[utf8]{inputenc}

% Default fixed font does not support bold face
\DeclareFixedFont{\ttb}{T1}{txtt}{bx}{n}{9} % for bold
\DeclareFixedFont{\ttm}{T1}{txtt}{m}{n}{9}  % for normal

\usepackage{listings}

% Python style for highlighting
%\newcommand\pythonstyle{\lstset{
%language=Python,
%basicstyle=\ttm,
%otherkeywords={self, none, return},             % Add keywords here
%keywordstyle=\ttb\color{deepblue},
%emph={MyClass,__init__},          % Custom highlighting
%emphstyle=\ttb\color{deepred},    % Custom highlighting style
%stringstyle=\color{deepgreen},
%frame=tb,                         % Any extra options here
%showstringspaces=false            %
%}}


% Python environment
%\lstnewenvironment{python}[1][]
%{
%$\pythonstyle
%\lstset{#1}
%}
%{}

% Python for external files
%\newcommand\pythonexternal[2][]{{
%\pythonstyle
%\lstinputlisting[#1]{#2}}}
%
%\lstnewenvironment{xml}
%{}
%{}

% Python for inline
%\newcommand\pythoninline[1]{{\pythonstyle\lstinline!#1!}}

%\def\DRAFT{} % Uncomment this if you want to see the notes people have been adding
% Comment command for developers (Should only be used under active development)
%\ifdefined\DRAFT
%  \newcommand{\nameLabeler}[3]{\textcolor{#2}{[[#1: #3]]}}
%\else
%  \newcommand{\nameLabeler}[3]{}
%\fi
% Commands for making the LaTeX a bit more uniform and cleaner
%\newcommand{\TODO}[1]    {\textcolor{red}{\textit{(#1)}}}
\newcommand{\xmlAttrRequired}[1] {\textcolor{red}{\textbf{\texttt{#1}}}}
\newcommand{\xmlAttr}[1] {\textcolor{cyan}{\textbf{\texttt{#1}}}}
\newcommand{\xmlNodeRequired}[1] {\textcolor{deepblue}{\textbf{\texttt{<#1>}}}}
\newcommand{\xmlNode}[1] {\textcolor{darkblue}{\textbf{\texttt{<#1>}}}}
\newcommand{\xmlString}[1] {\textcolor{black}{\textbf{\texttt{'#1'}}}}
\newcommand{\xmlDesc}[1] {\textbf{\textit{#1}}} % Maybe a misnomer, but I am
                                                % using this to detail the data
                                                % type and necessity of an XML
                                                % node or attribute,
                                                % xmlDesc = XML description
\newcommand{\default}[1]{~\\*\textit{Default: #1}}
\newcommand{\nb} {\textcolor{deepgreen}{\textbf{~Note:}}~}

% The bm package provides \bm for bold math fonts.  Apparently
% \boldsymbol, which I used to always use, is now considered
% obsolete.  Also, \boldsymbol doesn't even seem to work with
% the fonts used in this particular document...
\usepackage{bm}

% Define tensors to be in bold math font.
\newcommand{\tensor}[1]{{\bm{#1}}}

% Override the formatting used by \vec.  Instead of a little arrow
% over the letter, this creates a bold character.
\renewcommand{\vec}{\bm}

% Define unit vector notation.  If you don't override the
% behavior of \vec, you probably want to use the second one.
\newcommand{\unit}[1]{\hat{\bm{#1}}}

% Use this to refer to a single component of a unit vector.
\newcommand{\scalarunit}[1]{\hat{#1}}

% \toprule, \midrule, \bottomrule for tables
\usepackage{booktabs}

% \llbracket, \rrbracket
\usepackage{stmaryrd}

\usepackage{hyperref}
\hypersetup{
    colorlinks,
    citecolor=black,
    filecolor=black,
    linkcolor=black,
    urlcolor=black
}

% Compress lists of citations like [33,34,35,36,37] to [33-37]
\usepackage{cite}

% If you want to relax some of the SAND98-0730 requirements, use the "relax"
% option. It adds spaces and boldface in the table of contents, and does not
% force the page layout sizes.
% e.g. \documentclass[relax,12pt]{SANDreport}
%
% You can also use the "strict" option, which applies even more of the
% SAND98-0730 guidelines. It gets rid of section numbers which are often
% useful; e.g. \documentclass[strict]{SANDreport}

% The INLreport class uses \flushbottom formatting by default (since
% it's intended to be two-sided document).  \flushbottom causes
% additional space to be inserted both before and after paragraphs so
% that no matter how much text is actually available, it fills up the
% page from top to bottom.  My feeling is that \raggedbottom looks much
% better, primarily because most people will view the report
% electronically and not in a two-sided printed format where some argue
% \raggedbottom looks worse.  If we really want to have the original
% behavior, we can comment out this line...
\raggedbottom
\setcounter{secnumdepth}{5} % show 5 levels of subsection
\setcounter{tocdepth}{5} % include 5 levels of subsection in table of contents

% ---------------------------------------------------------------------------- %
%
% Set the title, author, and date
%
\title{'CashFlow' User Manual \\
        \large Economics plugin for RAVEN \\
}
%\author{%
%\begin{tabular}{c} Author 1 \\ University1 \\ Mail1 \\ \\
%Author 3 \\ University3 \\ Mail3 \end{tabular} \and
%\begin{tabular}{c} Author 2 \\ University2 \\ Mail2 \\ \\
%Author 4 \\ University4 \\ Mail4\\
%\end{tabular} }


\author{
 \\Aaron S. Epiney, Paul Talbot, Congjian Wang\\
}

% There is a "Printed" date on the title page of a SAND report, so
% the generic \date should [WorkingDir:]generally be empty.
\date{\today}

%\def\component#1{\texttt{#1}}

% ---------------------------------------------------------------------------- %
%\newcommand{\systemtau}{\tensor{\tau}_{\!\text{SUPG}}}

% Added by Sonat
%\usepackage{placeins}
%\usepackage{array}

%\newcolumntype{L}[1]{>{\raggedright\let\newline\\\arraybackslash\hspace{0pt}}m{#1}}
%\newcolumntype{C}[1]{>{\centering\let\newline\\\arraybackslash\hspace{0pt}}m{#1}}
%\newcolumntype{R}[1]{>{\raggedleft\let\newline\\\arraybackslash\hspace{0pt}}m{#1}}

% end added by Sonat
% ---------------------------------------------------------------------------- %
%
% Start the document
%

\begin{document}
    \maketitle

    % ------------------------------------------------------------------------ %
    % The table of contents and list of figures and tables
    % Comment out \listoffigures and \listoftables if there are no
    % figures or tables. Make sure this starts on an odd numbered page
    %
    \cleardoublepage		% TOC needs to start on an odd page
    \tableofcontents
    %\listoffigures
    %\listoftables
    % ---------------------------------------------------------------------- %

    % ---------------------------------------------------------------------- %
    % This is where the body of the report begins; usually with an Introduction
    %
    
\section{Installation}
\label{sec:Installation}

This section of the manual describes how to install the TEAL plug-in and use it from within RAVEN \cite{RAVEN}.

\subsection{Installing the plug-in}
The TEAL plugin is distributed with RAVEN. If you just want to use the plugin and are not interested in developing additional features or bug-fixes for it, just make sure that you have access to the plugin repository and you will automatically get it when installing RAVEN.
For more information, please refere to the RAVEN manual.

\subsection{Accessing the plug-in from within RAVEN}
The plugin can be accessed as a special subtype of the External model. The syntax is given in Listing \ref{lst:TEALfromRAVEN}.

\begin{lstlisting}[style=XML,morekeywords={anAttribute},caption=Call TEAL from RAVEN input., label=lst:TEALfromRAVEN]
<ExternalModel name="Cash_Flow" subType="TEAL.TEAL">
  <variables> Input and output variables needed by TEAL  </variables>
  <ExternalXML node="Economics" xmlToLoad="Cash_Flow_input.xml"/>
</ExternalModel>
\end{lstlisting}

\subsection{Running the plugin as a stand-alone python code}

In addition to accessing the plug-in from within RAVEN, it can also be run as a stand-alone python program. This is useful for example for testing. However, since TEAL is still a RAVEN plugin, RAVEN needs to be installed and the plugin needs to be in the plugin-folder for that to work.

Assuming RAVEN is installed and the plugin is in the proper directory (execution will generate an error if its not), one can run it using the command shown in Listing \ref{lst:TEALAsCode}.

\small
\begin{lstlisting}[caption=TEAL run as stand-alone python code, label=lst:TEALAsCode]
~/raven --> python plugins/TEAL/src/TEAL_ExtMode.py -h
usage: Cash_Flow.py [-h] -iXML inp_file -iINP inp_file -o out_file

Run RAVEN TEAL plugin as stand-alone code

arguments:
  -h, --help      show this help message and exit
  -iXML inp_file  XML TEAL input file name
  -iINP inp_file  TEAL input file name with the input
                  variable list
  -o out_file     Output file name
\end{lstlisting}
\normalsize

The XML TEAL input is the one that contains the component and cash flow definitions, i.e. the one in \xmlAttr{xmlToLoad} in Listing \ref{lst:TEALfromRAVEN}.

The TEAL input file contains the variables that RAVEN provides when using the TEAL as a plugin, i.e. the cash flow drivers and multipliers. Listing \ref{lst:TEALInputFile} shows an example of the TEAL input file for the cash flow definitions given in Listing \ref{lst:InputExample}.

\begin{lstlisting}[caption=TEAL run as stand-alone python code, label=lst:TEALInputFile]
Cfdriver1 5.5
multiplier1 1.0
Cfdriver2 10.8
multiplier2 2.0
\end{lstlisting}


    \section{CashFlow for RAVEN}
A generalized module (called CashFlow) for economic analysisi within RAVEN has been developed \cite{MSApril2017}. The module is able to compute
the NPV (Net Present Value), the IRR (Internal Rate of Return) and the PI (Profitability Index). Furthermore, it is possible to
do an NPV, IRR or PI search, i.e. CashFlow will compute a multiplicative value (for example the production cost) so that the
NPV, IRR or PI has a desired value (for details see \ref{subsec:NPV_search}, NPV\_search). This CashFlow module has been written using the script language Python.
The Python code can be used as an “external model” in RAVEN (for installation and usage instructions, see \ref{sec:Installation}).

The input of CashFlow is an XML file. An example of the input structure is given in Listing \ref{lst:InputExample}. The following section will discuss the
 different keywords in the input and describe how they are used in the CashFlow module.

\begin{lstlisting}[style=XML,morekeywords={anAttribute},caption=Economics input example., label=lst:InputExample]
<Economics verbosity='0'>
    <Global>
        <Indicator name='IRR,NPV_search,NPV' target='0'>
            Cfname1
            Cfname2
            ...
        </Indicator>
        <DiscountRate>0.08</DiscountRate>
        <tax>0.392</tax>
        <inflation>0.04</inflation>
        <ProjectTime>100</ProjectTime> <!-- optional -->
    </Global>

    <Component name='Component1'>
        <Life_time>20</Life_time>
        <StartTime>10</StartTime> <!-- optional -->
        <Repetitions>3</Repetitions> <!-- optional -->
        <tax>0.3</tax> <!-- optional -->
        <inflation>0.07</inflation> <!-- optional -->
	<CashFlows>
	    <Capex name='Cfname1' tax='false' inflation='none' multiply='multiplier1' mult_target='false'>
    	        <driver>Cfdriver1</driver>
                <alpha>-4000000000</alpha>
                <reference>1000000000</reference>
                <X>1.0</X>
            </Capex>

            <Recurring name='Cfname2' tax='false' inflation='none' multiply='multiplier2' mult_target='true'>
                ...
            </Recurring>
            ...
	</CashFlows>
    </Component>

    <Component name='Component2'>
        ...
    </Component>
    ...
</Economics>
\end{lstlisting}

As one can see, all the specifications of the CashFlow module are given in the \xmlNode{Economics} block. The block accepts an attribute called \xmlAttr{verbosity},
which can range from 0 to 100, 0 meaning maximum debug verbosity and 100 meaning errors only. Setting the verbosity to 50 will output (in addition to errors) the NPV, IRR, PI or NPV\_mult. Inside the \xmlNode{Economics} block, there are two types of blocks: \xmlNode{Global} and \xmlNode{Component}.

\subsection{\xmlNode{Global}}
Exactly one \xmlNode{Global} block has to be provided. The \xmlNode{Global} block does not have any attributes. The following sub-blocks can be given in the \xmlNode{Global} block:

\begin{enumerate}
\item[\xmlNode{Indicator}] List of cash flows considered in the computation of the economic indicator. See later for the definition
 of the cash flows. Only cash flows listed here are considered, additional cash flows defined, but not listed are ignored.
\xmlNode{Indicator} can have two attributes:
  \begin{enumerate}
  \item[\xmlAttr{name}] The names of the economic indicators that should be computed. So far, \textbf{'NPV'}, \textbf{'NPV\_search'}, \textbf{'IRR'} and \textbf{'PI'} are supported. More than one indicator can be asked for.
The \xmlAttr{name} attribute can contain a comma-separated list as shown in the example in Listing \ref{lst:InputExample}.

\textbf{NPV}: computes the NPV according to Eq. \ref{eq:NPV}.
\begin{equation}\label{eq:NPV}
NPV=\sum_{y=0}^{N}\frac{CF_{y}}{(1+DiscountRate)^{y}}
\end{equation}

The sum runs over the years $y=0$ to $N$. The net cash flows $CF_{y}$ are the sum of all cash flows defined in the indicator block (see later for how to define these cash flows).
$N$ is the least common multiple (LCM) of all component life times involved. This guarantees that the NPV is computed for a time span so that all components reach their end of life in the same year.
The individual component cash flows are repeated until the LCM is reached. For example, lets assume the calculation involves two components \textit{Component1} and \textit{Component2}
 with life times of 60 years and 40 years respectively. $N$ will be 120 years where 2 successive \textit{Component1} and 3 successive \textit{Component2} will be build. For every ‘building year’,
the cash flow for the last year (of the old component) and the year zero (for the newly built component) will be summed. Table \ref{tbl:cashflows} shows an example for illustration.
The variable sent back to RAVEN, i.e. what needs to be added to the output data object is 'NPV'.

\begin{table}[]
\centering
\caption{Example cash flows for NPV calculation.}
\label{tbl:cashflows}
\begin{tabular}{ll|l|l|l|l|l|ll}
\cline{3-4} \cline{6-7}
                           &  & \multicolumn{2}{l|}{Compo 1}                                                                                                   &  & \multicolumn{2}{l|}{Compo 2}                                                                                                     &                       &                                                                                                       \\ \cline{1-1} \cline{3-4} \cline{6-7} \cline{9-9}
\multicolumn{1}{|l|}{Year} &  & \begin{tabular}[c]{@{}l@{}}Comp. \\ lifetime\end{tabular} & \begin{tabular}[c]{@{}l@{}}Cash Flow\\ (year)\end{tabular}         &  & \begin{tabular}[c]{@{}l@{}}Compo. \\ Lifetime\end{tabular} & \begin{tabular}[c]{@{}l@{}}  Cash Flow \\ (year) \end{tabular}      & \multicolumn{1}{l|}{} & \multicolumn{1}{l|}{\begin{tabular}[c]{@{}l@{}}Total Net Cash flow \\ ($CF_{y}$)         \end{tabular}}   \\ \cline{1-1} \cline{3-4} \cline{6-7} \cline{9-9}
\multicolumn{1}{|l|}{0}    &  & 0                                                         & $CF^{comp1}_{0}$                                                          &  & 0                                                          & $CF^{comp2}_{0}$                                                           & \multicolumn{1}{l|}{} & \multicolumn{1}{l|}{$CF^{comp1}_{0} + CF^{comp2}_{0}$ }                                     \\ \cline{1-1} \cline{3-4} \cline{6-7} \cline{9-9}
\multicolumn{1}{|l|}{1}    &  & 1                                                         & $CF^{comp1}_{1}$                                                          &  & 1                                                          & $CF^{comp2}_{1}$                                                           & \multicolumn{1}{l|}{} & \multicolumn{1}{l|}{$CF^{comp1}_{1} + CF^{comp2}_{1}$ }                                     \\ \cline{1-1} \cline{3-4} \cline{6-7} \cline{9-9}
\multicolumn{1}{|l|}{…}    &  &                                                           &                                                                    &  &                                                            &                                                                     & \multicolumn{1}{l|}{} & \multicolumn{1}{l|}{}                                                                                 \\ \cline{1-1} \cline{3-4} \cline{6-7} \cline{9-9}
\multicolumn{1}{|l|}{39}   &  & 39                                                        & $CF^{comp1}_{39}$                                                         &  & 39                                                         & $CF^{comp2}_{39}$                                                          & \multicolumn{1}{l|}{} & \multicolumn{1}{l|}{$CF^{comp1}_{39} + CF^{comp2}_{39}$ }                                   \\ \cline{1-1} \cline{3-4} \cline{6-7} \cline{9-9}
\multicolumn{1}{|l|}{40}   &  & 40                                                        & $CF^{comp1}_{40}$                                                         &  & 40 and 0                                                   & \begin{tabular}[c]{@{}l@{}}$CF^{comp2}_{40}$  \\ $+ CF^{comp2}_{0}$ \end{tabular} & \multicolumn{1}{l|}{} & \multicolumn{1}{l|}{\begin{tabular}[c]{@{}l@{}}$CF^{comp1}_{40} + CF^{comp2}_{40}$ \\  $+ CF^{comp2}_{0}$ \end{tabular}} \\ \cline{1-1} \cline{3-4} \cline{6-7} \cline{9-9}
\multicolumn{1}{|l|}{41}   &  & 41                                                        & $CF^{comp1}_{41}$                                                         &  & 1                                                          & $CF^{comp2}_{1}$                                                           & \multicolumn{1}{l|}{} & \multicolumn{1}{l|}{$CF^{comp1}_{41} + CF^{comp2}_{1}$ }                                  \\ \cline{1-1} \cline{3-4} \cline{6-7} \cline{9-9}
\multicolumn{1}{|l|}{…}    &  &                                                           &                                                                    &  &                                                            &                                                                     & \multicolumn{1}{l|}{} & \multicolumn{1}{l|}{}                                                                                 \\ \cline{1-1} \cline{3-4} \cline{6-7} \cline{9-9}
\multicolumn{1}{|l|}{59}   &  & 59                                                        & $CF^{comp1}_{59}$                                                         &  & 19                                                         & $CF^{comp2}_{19}$                                                          & \multicolumn{1}{l|}{} & \multicolumn{1}{l|}{$CF^{comp1}_{59} + CF^{comp2}_{19}$ }                                 \\ \cline{1-1} \cline{3-4} \cline{6-7} \cline{9-9}
\multicolumn{1}{|l|}{60}   &  & 60 and 0                                                  & \begin{tabular}[c]{@{}l@{}}$CF^{comp1}_{60}$ \\ $+ CF^{comp1}_{0}$ \end{tabular} &  & 20                                                         & $CF^{comp2}_{20}$                                                          & \multicolumn{1}{l|}{} & \multicolumn{1}{l|}{\begin{tabular}[c]{@{}l@{}}$CF^{comp1}_{60} + CF^{comp1}_{0}$ \\ $+ CF^{comp2}_{20}$ \end{tabular}} \\ \cline{1-1} \cline{3-4} \cline{6-7} \cline{9-9}
\multicolumn{1}{|l|}{61}   &  & 1                                                         & $CF^{comp1}_{1}$                                                          &  & 21                                                         & $CF^{comp2}_{21}$                                                          & \multicolumn{1}{l|}{} & \multicolumn{1}{l|}{$CF^{comp1}_{1} + CF^{comp2}_{21}$ }                                                          \\ \cline{1-1} \cline{3-4} \cline{6-7} \cline{9-9}
\multicolumn{1}{|l|}{…}    &  &                                                           &                                                                    &  &                                                            &                                                                     & \multicolumn{1}{l|}{} & \multicolumn{1}{l|}{}                                                                                 \\ \cline{1-1} \cline{3-4} \cline{6-7} \cline{9-9}
\multicolumn{1}{|l|}{79}   &  & 19                                                        & $CF^{comp1}_{19}$                                                         &  & 39                                                         & $CF^{comp2}_{39}$                                                          & \multicolumn{1}{l|}{} & \multicolumn{1}{l|}{$CF^{comp1}_{19} + CF^{comp2}_{39}$ }                                                         \\ \cline{1-1} \cline{3-4} \cline{6-7} \cline{9-9}
\multicolumn{1}{|l|}{80}   &  & 20                                                        & $CF^{comp1}_{20}$                                                         &  & 40 and 0                                                   & \begin{tabular}[c]{@{}l@{}}$CF^{comp2}_{40}$ \\  $+ CF^{comp2}_{0}$ \end{tabular} & \multicolumn{1}{l|}{} & \multicolumn{1}{l|}{\begin{tabular}[c]{@{}l@{}}$CF^{comp1}_{20} + CF^{comp2}_{40}$ \\ $+ CF^{comp2}_{0}$ \end{tabular}}  \\ \cline{1-1} \cline{3-4} \cline{6-7} \cline{9-9}
\multicolumn{1}{|l|}{81}   &  & 21                                                        & $CF^{comp1}_{21}$                                                         &  & 1                                                          & $CF^{comp2}_{1}$                                                           & \multicolumn{1}{l|}{} & \multicolumn{1}{l|}{$CF^{comp1}_{21} + CF^{comp2}_{1}$ }                                                          \\ \cline{1-1} \cline{3-4} \cline{6-7} \cline{9-9}
\multicolumn{1}{|l|}{…}    &  &                                                           &                                                                    &  &                                                            &                                                                     & \multicolumn{1}{l|}{} & \multicolumn{1}{l|}{}                                                                                 \\ \cline{1-1} \cline{3-4} \cline{6-7} \cline{9-9}
\multicolumn{1}{|l|}{119}  &  & 59                                                        & $CF^{comp1}_{59}$                                                         &  & 39                                                         & $CF^{comp2}_{39}$                                                          & \multicolumn{1}{l|}{} & \multicolumn{1}{l|}{$CF^{comp1}_{59} + CF^{comp2}_{39}$ }                                                         \\ \cline{1-1} \cline{3-4} \cline{6-7} \cline{9-9}
\multicolumn{1}{|l|}{120}  &  & 60                                                        & $CF^{comp1}_{60}$                                                         &  & 40                                                         & $CF^{comp2}_{40}$                                                          & \multicolumn{1}{l|}{} & \multicolumn{1}{l|}{$CF^{comp1}_{60} + CF^{comp2}_{40}$}                                                         \\ \cline{1-1} \cline{3-4} \cline{6-7} \cline{9-9}
\end{tabular}
\end{table}

\textbf{PI}: computes the PI according to Eq. \ref{eq:PI}.
\begin{equation}\label{eq:PI}
PI=\frac{NPV}{Initial\_investment}
\end{equation}
where the NPV is calculated as explained above and the $Initial\_investment$ is the Total Net Cash flow at year zero, i.e. $CF_{0}$ in the example above.
The variable sent back to RAVEN, i.e. what needs to be added to the output data object is 'PI'.

\textbf{IRR}: computes the IRR according to Eq. \ref{eq:IRR}.
\begin{equation}\label{eq:IRR}
0=\sum_{y=0}^{N}\frac{CF_{y}}{(1+IRR)^{y}}
\end{equation}
Same as for the NPV, the sum runs over the years $y=0$ to $N$. The net cash flows $CF_{y}$ are the sum of all cash flows defined in the indicator block
(see explanation of NPV above for details). $N$ is the least common multiple (LCM) of all component life times involved.
The variable sent back to RAVEN, i.e. what needs to be added to the output data object is 'IRR'.

\textbf{NPV\_search}: The NPV search finds a multiplier '$x$' that multiplies some of the cash flows, so that the NPV has a desired value (defined by the \xmlAttr{target} attribute). The equation solved is shown in Eq. \ref{eq:NPV_search}.
\label{subsec:NPV_search}
\begin{equation}\label{eq:NPV_search}
'target'=\sum_{y=0}^{N}\frac{CF^{dep\_on\_x}_{y}}{(1+DiscountRate)^{y}}x + \sum_{y=0}^{N}\frac{CF^{not\_dep\_on\_x}_{y}}{(1+DiscountRate)^{y}}
\end{equation}

The cash flows that multiply ‘$x$’ have to have the \xmlAttr{mult\_target} attribute equal \textbf{'true'} (see later in cash flow definition). This functionality can be used for
 example to find a commodity price so that the NPV is zero. In this case, the \xmlAttr{target} will be set to \textbf{'0'} and all cash flows that depend (linearly) on the price will take
 \xmlAttr{mult\_target}$=$\textbf{'true'}, i.e. for example the revenue, while cash flows that do not depend on the price will have \xmlAttr{mult\_target}$=$\textbf{'false'}, i.e. for example the capital cost.
The variable sent back to RAVEN, i.e. what needs to be added to the output data object is 'NPV\_mult'.

\textbf{Note on IRR and PI search}: It should be noted that although the only search keyword allowed in \xmlAttr{name} is \textbf{NPV\_search}, it is possible to perform IRR and PI searches as well.

  \begin{itemize}
  \item To do an IRR search, the DiscountRate is set to the desired IRR and a NPV search with the target of ‘0’ is performed.
  \item To perform a PI search, an NPV search can be performed where the target PI is multiplied with the initial investment.
  \end{itemize}


  \item[\xmlAttr{Target}] Target value for the NPV search, i.e. \textbf{'0'} will look for ‘$x$’ so that $NPV(x) = 0$.

  \end{enumerate}

\item[\xmlNode{DiscountRate}] The discount rate used to compute the NPV and PI. Not used for the computation of the IRR (although it must be input).
\item[\xmlNode{tax}] The standard tax rate used to compute the taxes if no other tax rate is specified in the componet blocks. This is a required input. If a tax rate is specified inside a component block, the componet will use that tax rate. If no tax rate is specified in a component, this standard tax rate is used for the component. See later in the definition of the cash flows for more details how the tax rate is used.
\item[\xmlNode{inflation}] The standard inflation rate used to compute the inflation if no other inflation rate is specified in the componet blocks. This is a required input. If a inflation rate is specified inside a component block, the componet will use that inflation rate. If no inflation rate is specified in a component, this standard inflation rate is used for the component. See later in the definition of the cash flows for more details how the inflation rate is used.

\item[\xmlNode{ProjectTime}] This is a optional input. If it is included in the input, the global project time is not the LCM of all components (see \xmlNode{Indicator} attribute \xmlAttr{name} for more information), but the time indicated here.

\end{enumerate}


\subsection{\xmlNode{Component}}

The user can define as many \xmlNode{Component} blocks as needed. A component is typically a part of the system that has the same lifetime and
the same cash flows, i.e. for example a gas turbine, a battery or a nuclear plant. Each component needs to have a \xmlAttr{name} attribute that is unique.
Each \xmlNode{Component} has to have one \xmlNode{Life\_time} block and as many \xmlNode{CashFlow} blocks as needed.

\begin{enumerate}
  \item[\xmlNode{Life\_time}] The lifetime of the component in years. This is used to compute the least common multiple (LCM) of all components involved in the
    computation of the economics indicator. For more details see NPV, IRR and PI explanations above.
  \item[\xmlNode{tax}] This is a optional input. If the tax rate is specified here, i.e. inside the component block, the componet will use this tax rate.
    If no tax rate is specified in the component, the standard tax rate from the \xmlNode{Global} block is used for the component.
  \item[\xmlNode{inflation}] This is a optional input. If the inflation rate is specified here, i.e. inside the component block,
    the componet will use this inflation rate. If no inflation rate is specified in the component, the standard inflation rate from the \xmlNode{Global}
    block is used for the component.
  \item[\xmlNode{StartTime}] This is a optional input. If this input is specified for one or more components, the \xmlNode{Global}
    input \xmlNode{ProjectTime} is required. This input specifies the year in which this component is going to be build for the first time,
    i.e. is going to be included in the cash flows. The default is 0 and the componet is build at the start of the project, i.e. at project year 0.
    For example, if the \xmlNode{ProjectTime} is 100 years, and for this component, the \xmlNode{StartTime} is 20 years, the cash flows for this
    component are going to be zero for years 0 to 19 of the project. Year 20 of the project will be year 0 of this component and so on
    (project year 21 will be component year 1 etc.).
  \item[\xmlNode{Repetitions}] This is a optional input. If this input is specified for one or more components, the \xmlNode{Global}
    input \xmlNode{ProjectTime} is required. This input specifies the number of times this component is going to be rebuilt. The default is 0,
    which indicates that the component is going to be rebuild indefinitely until the project end (\xmlNode{ProjectTime}) is reached.
    Lets assume the \xmlNode{ProjectTime} is 100 years, and the component \xmlNode{Life\_time} is 20 years. Specifying 3 repetitions of this
    component will build 3 components in succession, at years 0, 20 and 40. For years 61 to 100 of the project, the cash flows for this component will be zero.

  \item[\xmlNode{CashFlows}] The user can define any number of 'cash flows' for a component. Each cash flow is of the form given in
    Eq. \ref{eq:CF} where $y$ is the year from 0 (capital investment) to the end of the \xmlNode{Life\_time} of the component.
    \begin{equation}\label{eq:CF}
    CF_{y}=mult\cdot\alpha_{y}\left ( \frac{driver_{y}}{ref} \right )^{X}
    \end{equation}

  The \xmlNode{CashFlows} currently can accept the following subnodes:


  \begin{enumerate}
    \item[\xmlAttr{name}] The name of the Cash flow. Has to be unique across all components. This is the name that can be listed in the
      \xmlNode{Indicator} node of the \xmlNode{Global} block.
    \item[\xmlAttr{driver}] The $driver$ in Eq. \ref{eq:CF} of the cash flow. This can be any variable passed in from RAVEN or the name
      of another cash flow. If it is passed in from RAVEN, it has to be either a scalar or a vector with length \xmlNode{Life\_time} + 1.
      If its a scalar, all $driver_{y}$ in Eq. \ref{eq:CF}  are the same for all years of the project life. If it is a vecor instead, each
      year of the project \xmlNode{Life\_time} will have its corresponding value for the driver. If the driver is another
      cash flow, the project \xmlNode{Life\_time} of the component to which the driving cash flow belongs has to be the same than the project
      \xmlNode{Life\_time} of the component to which this
      cash flow belongs. No loops of cash flows are allowed. The code will error out if there are loops of cash flows, i.e. A is the driver
      of B and B the driver of A.
    \item[\xmlAttr{tax}] Can be \textbf{true} or \textbf{false}. If it is \textbf{true}, the cash flow is multiplied by $(1-tax)$, where tax
      is the tax rate given in \xmlNode{tax} in the \xmlNode{Global}
      block. As an example, the cash flow of \textit{comp2} for year 119 in Listing \ref{lst:InputExample} would become $CF^{comp2}_{39}(1-tax)$.
      If a cash flow with \xmlAttr{tax}$=$\textbf{true} is the driver of another cash flow, the cash flow without the tax is used as driver for the new cash flow.
      The limitation of having a global tax rate will be lifted in future version of the CashFlow module. It is planned to have the possibility to
      input different tax rates for each component, since they might be in different tax regions.
    \item[\xmlAttr{inflation}] Can be \textbf{real, nominal} or \textbf{none}. If it is \textbf{real}, the cash flow is multiplied by
      $(1+inflation)^{-y}$. If it is \textbf{nominal}, the cash flow is multiplied by $(1+inflation)^y$.
      In both cases, inflation is given by \xmlNode{inflation} in the \xmlNode{Global} block. Furthermore, $y$ goes from year 0 (capital investment)
      to the LCM of all component lifetimes.
      This means that the cash flows as expressed in Listing \ref{lst:InputExample} are multiplied with the infloation seen from today, i.e. the cash
      flow for \textit{comp2} for year 119 assuming it includes \textbf{real}
      inflation would be $CF^{comp2}_{39}(1+inflation)^{-119}$
      If a cash flow with \xmlAttr{inflation} equal \textbf{real} or \textbf{nominal} is the driver of another cash flow, the cash flow without
      the inflation is used as driver for the new cash flow.
    \item[\xmlAttr{multiply}] This is an optional attribute. This can be the name of any scalar variable passed in from RAVEN. This number
      is $mult$ in Eq. \ref{eq:CF} that multiplies the cash flow.
    \item[\xmlAttr{mult\_target}] Can be \textbf{true} or \textbf{false}. If \textbf{true}, it means that this cash flow multiplies
      the search variable '$x$' as explained in the NPV\_search option above.
      If the NPV\_search option is used, al least one cash flow has to have \xmlAttr{mult\_target}$=$\textbf{true}.
    \item[\xmlNode{alpha}] $\alpha_{y}$ multiplier of the cash flow (see Eq. \ref{eq:CF}). Exactly \xmlNode{Life\_time}$ + 1$
      values are expected. One for $y=0$ to $y=$\xmlNode{Life\_time}.
    \item[\xmlNode{reference}] The $ref$ value of the cash flow (see Eq. \ref{eq:CF}).
    \item[\xmlNode{X}] The $X$ exponent (economy of scale factor) of the cash flow (see Eq. \ref{eq:CF}).
  \end{enumerate}
\end{enumerate}

An example of a cash flow is shown in Listing \ref{lst:CashFlowExample}. In the example, a cash flow called CAPEX is defined.
In the example, the capital expenditure for a reference plant of capacity
 1'000'000'000 W is \$4 billion. The driver of this cash flow is the actual plant capacity in Watts. Building this plat has some economy
 of scale, so that a plant double the size does cost
less than double the money ($X=0.64$). The example assums an overnight building cost of the reactor, i.e. $\alpha$ is zero for
all years of the lifetime of the reactor except year zero, which is the construction year.

\begin{lstlisting}[style=XML,morekeywords={anAttribute},caption=CashFlow input example., label=lst:CashFlowExample]
<Capex name='CAPEX' driver='Plant_capacity' tax='false' inflation='none'>
    <driver>Plant_capacity</driver>
    <alpha>-4000000000</alpha>
    <reference>1000000000</reference>
    <X>0.64</X>
</Capex>
\end{lstlisting}


    \section*{Document Version Information}
    This document has been compiled using the following version of the plug-in git repository:
    \newline
    f8a46c15f16e3c95179586e647f807b684fb36d3 Paul W. Talbot\\Tue, 29 Sep 2020 13:20:59 -0600


    % ---------------------------------------------------------------------- %
    % References
    %
    \clearpage
    % If hyperref is included, then \phantomsection is already defined.
    % If not, we need to define it.
    \providecommand*{\phantomsection}{}
    \phantomsection
    \addcontentsline{toc}{section}{References}
    \bibliographystyle{ieeetr}
    \bibliography{user_manual}


    % ---------------------------------------------------------------------- %

\end{document}
