\section{Introduction}
\subsection{System Purpose}

The \textbf{\textit{TEAL}} plug-in is a generalized module for economic analysis within RAVEN.
\\RAVEN is a flexible and multi-purpose uncertainty quantification (UQ), regression analysis, probabilistic risk assessment 
(PRA), data analysis and model optimization software.  Depending on the tasks to be accomplished and on the 
probabilistic
 characterization of the problem, RAVEN perturbs (Monte-Carlo, Latin hyper-cube, reliability surface search, etc.) the
 response of the system under consideration by altering its own parameters. 
 The data generated by the sampling process is analyzed using classical statistical
 and more advanced data mining approaches. RAVEN also manages the parallel dispatching (i.e. both on
 desktop/workstation and large High-Performance Computing machines) of the software representing the physical 
 model.
 For more information about the RAVEN software, see ~\cite{RAVENuserManual} and the RAVEN website (\url{https://raven.inl.gov})
\\The  \textbf{\textit{TEAL}} module is a dedicated module of RAVEN and one of the first developed RAVEN plug-ins.
A RAVEN plug-in is a software/module/library that has been developed to be linked to RAVEN at run-time, using the RAVEN APIs.


\subsection{System Scope}

The \textbf{\textit{TEAL}} plug-in’s scope is to provide a set of capabilities to compute the NPV (Net Present Value), the IRR
(Internal Rate of Return) and the PI (Profitability Index). Furthermore, it is possible to do an NPV, IRR or PI search, i.e. TEAL will
compute a multiplicative value (for example the production cost) so that the NPV, IRR or PI has a desired value.

 The main objective of the module (in conjunction with the RAVEN software) is to assist the engineer/user to:
\begin{itemize}
  \item identify/compute the main economic figures of merit characterizing the operation of a complex system;
  \item estimate the likelihood of undesired economic outcomes (economic risk analysis);
  \item identify main drivers to act on for reducing impact/consequences of adverse economic trends of the 
         system under analysis.
\end{itemize}

In other words, the  \textbf{\textit{TEAL}} plug-in (driven by RAVEN) is aimed to be employed for:
\begin{itemize}
  \item Economic Analysis;
  \item Sensitivity Analysis / Regression Analysis of Economic Figures of Merit;
  \item Economic Risk Analysis;
  \item Economic Optimization.
\end{itemize}


\subsection{User Characteristics}

The users of the \textbf{\textit{TEAL}} plug-in are expected to be part of any of the
following categories:
\begin{itemize}
  \item \textbf{Core developers (TEAL core team)}: These are the developers of the \textbf{\textit{TEAL}}  plug-in. They will be responsible for following
    and enforcing the appropriate software development standards. They will be responsible for designing, implementing and 
    maintaining the plug-in.
  \item \textbf{External developers}: A Scientist or Engineer that utilizes the \textbf{\textit{TEAL}}  plug-in and wants to extend its
  capabilities.This user will typically have a background in modeling and 
simulation techniques and/or numerical and economic analysis but may only have a limited skill-set when it comes to object-oriented 
coding, C++/Python languages.
  \item \textbf{Analysts}:  These are users that will run the plug-in (in conjunction with RAVEN) and perform various analysis on the 
  simulations they perform. These users may interact with developers of the system requesting new features and reporting bugs found 
  and will typically make heavy use of the input file format.
\end{itemize}
